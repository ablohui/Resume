\documentclass{resume}
\usepackage[usenames,dvipsnames]{xcolor}
\usepackage{hyperref}
\definecolor{linkcolour}{rgb}{0,0.2,0.6}  %蓝色
\hypersetup{colorlinks,breaklinks,urlcolor=linkcolour, linkcolor=linkcolour}	
\usepackage{amsmath, bm}

\newcommand{\myThemeColor}{OliveGreen}
\newcommand{\beforetitleColor}{lightgray}
\newcommand{\CVtitle}[2]
{\normalsize \textbf{\color{\beforetitleColor} #1} \textbf{\color{\myThemeColor} #2} \vspace{0.2cm} \color{\beforetitleColor}\hrule \color{black} }

\newcommand{\CVitem}[2]
{ \vspace{0.2cm}  \footnotesize \textbf{#1} \par\vspace{0.1cm} \footnotesize #2 \vspace{0.2cm} }
\newcommand{\CVitemenglish}[2]
{ \vspace{0.2cm}  \footnotesize\boldsymbol{#1} \par\vspace{0.1cm} \footnotesize#2 \vspace{0.2cm} }
\newcommand{\Educationitem}[5]
{	
	\begin{tabularx}{\linewidth}{@{}m{0.25\linewidth}XX}\footnotesize
		\bf{\color{\myThemeColor}#1} &\footnotesize \textbf{#2} &\footnotesize #3 \\
		 \qquad  &\footnotesize #4 &\footnotesize #5	 	
	\end{tabularx}
}
\newcommand{\projectitem}[7]{
	\begin{tabularx}{\linewidth}{@{}m{0.25\linewidth} @{}m{0.55\linewidth} @{}m{0.2\linewidth} }\footnotesize		
		\bf{\color{\myThemeColor}#1}    & \footnotesize \textbf{#2}    & \footnotesize \textbf{#3}     	
	\end{tabularx}
	\begin{tabularx}{\linewidth}{@{}m{0.22\linewidth} X}\footnotesize
		\textbf{#4}  & \footnotesize#5 \\
		\footnotesize\textbf{#6}  & \footnotesize#7
	\end{tabularx}
	
}
\newcommand{\paperitem}[6]{
			\begin{tabularx}{\linewidth}{@{}m{0.22\linewidth} X}\footnotesize
				
				\bf{\color{\myThemeColor}#1}    & \footnotesize \bf{#2}   \\
				\footnotesize \textbf{#3} & \footnotesize #4 \\
				\footnotesize \textbf{#5} & \footnotesize #6 	
			\end{tabularx}
		
}
\begin{document}
	
	\fontfamily{ppl}\selectfont
	
	\noindent	
	\begin{tabularx}{\linewidth}{X}
		\LARGE{\textbf{\color{\myThemeColor}X X}} \Large{\textbf{\color{\beforetitleColor} XX}}  		
	\end{tabularx}
	%\vspace{-1cm}
	\color{\beforetitleColor}\rule{\textwidth}{0.1mm}
	%\vspace{0.5cm}
	\begin{center}
		\begin{tabularx}{\linewidth}{@{}m{0.25\textwidth} m{0.72\textwidth}@{}}
			
			{	
				\vspace{-0.5cm}	
				\centering\includegraphics[width=3cm]{cauldron.png}	\\		
				\raggedleft
				\vspace{0.5cm}
				\CVtitle{//}{个人信息} \\
				\vspace{0.1cm}
				\CVitem{姓名}{XX} \\
				\CVitem{出生年月}{XXXX.XX} \\
				\CVitem{籍贯}{XXXXXX} \\
				\CVitem{电话}{XXXXXX} \\
				\CVitemenglish{Email}{\href{XX@XX}{XX@XX}  }\\
				\CVitemenglish{GitHub}{\href{XXX}{XXX}  }\\
				
				\vspace{0.3cm}
				\CVtitle{//}{所获荣誉} \\
				
				\begin{itemize}
					\footnotesize
					\item XXXXXXXXXXXXXXX;
					\item XXXXXXXXXXXX;
					\item XXXXXXXXXXXX;
					\item XXXXXX;
					\item XXXXXXXXXX;
				\end{itemize}
				
				
			} &
			{
				\raggedright
				
				\CVtitle{//}{教育经历 } \\
					\Educationitem{2019.09 - 至今}{XX大学}{XX硕士}{XX}{XX排名:XX} 
					\Educationitem{2015.09 - 2019.06}{XX大学}{XX学士}{XX}{XX排名:XX} 	
				\vspace{0.1cm}			
				\CVtitle{//}{科研经历 } \\
					\projectitem{2021.01 - 至今}{XXXXXXX}{个人科研项目}{项目介绍:}{XXXXXXXXXXXXXXXXXXXXXXXXXXXXXXX}{主要技术:}{XXX,XX模型,XXXX}\\
					\paperitem{2020.08 - 2021.01}{XXXXXXXXXXXXXXXXXXXXXX}{论文介绍:}{XXXXXXXXXXXXXXXXXXXXXXXXXXXXXXXXXXXXXXXXXXXXXXXXXXXX}{主要技术:}{XXXX,XXXX,XXXX}\\
					\projectitem{2019.09 - 2020.06}{XXXXXXX}{XXX}{项目介绍:}{XXXXXXXXXXXXXXXXXXXXXXXXXXXXXXXXXXXXXXXXXXXXXXXXXXXXXXXXXXXXXXXXXX}{主要技术: }{XXXX、XXXXX}\\
				\vspace{0.2cm}	
				  \CVtitle{//}{项目经历 } \\
					\projectitem{2018.02 - 2018.06}{XXXXXX}{个人项目}{项目介绍:}{XXXXXXXXXXXXXXXXXXXXXXXXXXXXXXXXXXXXXXX}{主要技术}{XXX、XXX、XXXX}\\
					\projectitem{2016.11 - 2017.11}{XXXXX}{项目负责人}{项目介绍:}{XXXXXXXXXXXXXXXXXXXXXXXXXXXXXXXXXXXXXXXXXXXXXXXX}{主要技术:}{XXXX、XXXX}\\
				\vspace{0.2cm}
				\CVtitle{//}{专业技能 } \\
				
				\footnotesize
				\begin{itemize}					
					\item 熟悉Java和Python,了解C/C++;
					\item 有前后端开发经验,有XXXXXXXXXX等基础,对XXXXXXXXX等框架均有使用经验;
					\item 熟悉常用的机器学习及深度学习算法,对主流框架(XXXXXXXX等)均有使用经验,并对XXXXXXX有它一定的了解;
					\item 熟悉常用数据库(XXXXXXXX等)以及Linux服务器基本配置;			
				\end{itemize}
			}
		\end{tabularx}
	\end{center}
	
\end{document}